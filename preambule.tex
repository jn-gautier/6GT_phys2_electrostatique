%importation des packages
\documentclass[11pt,twoside,DIV=calc]{scrbook}
\usepackage{scrlayer-scrpage} %équivalent de fancyhdr pour kommascript
\usepackage[french]{babel}
\usepackage[babel=true]{csquotes} % csquotes va utiliser la langue définie dans babel
\usepackage[shortlabels]{enumitem} %permet de faire \begin{enumerate}[a)]
\usepackage{tabularx}
\usepackage{amsmath} %formules mathématiques
\usepackage{amssymb} %symboles mathématiques
\usepackage{amsfonts} %polices mathématiques
\usepackage{graphicx}
\usepackage{wrapfig}%image à doite ou a gauche du texte
\usepackage{tikz}
\usepackage{multido}%pour la macro pointillés
\usepackage[breakable]{tcolorbox}

\usepackage{pstricks}%pour dessiner les champs électriques
\usepackage{pst-electricfield}%pour dessiner les champs électriques

%hyperliens
\usepackage{hyperref}
\hypersetup{
  colorlinks,
  citecolor=black,
  filecolor=black,
  linkcolor=black,
  urlcolor=black
}


%police
%\usepackage{libertine} %la police
\usepackage[T1]{fontenc}

%cadre
\usepackage{framed}

%graphique tikz
\usepackage{tkz-euclide}
\usepackage{tkz-fct}

%chemin vers les images
\graphicspath{ {./images/} }

%exercices xsim
\usepackage[clear-aux]{xsim}
\DeclareExerciseTagging{difficulty}
\xsimsetup{
  path=xsim,
  load-style = layouts ,
  exercise/template = runin ,
  solution/template = runin,
  difficulty={*,**,***},
}



%math
\everymath{\displaystyle} %pour que les équations soient bien présentées


%mise en page générale
\KOMAoptions{DIV=last}
\usepackage{setspace}%pour regler l'interligne
\onehalfspacing %interligne1.5
\setlength{\parindent}{0pt} %pas de retrait en début de paragraphe
\usepackage{caption}%mise en forme des légendes des figures
\addtokomafont{caption}{\scriptsize \itshape}
\addtokomafont{captionlabel}{\scriptsize \itshape}

%pieds de page utiliser scrlayer-scrpage
\setkomafont{pageheadfoot}{\small}
\lohead{} \rohead{} \cohead{}
\lofoot{Électromagnétisme - physique(2h) - 6GT}
\rofoot{Page \thepage}
\cofoot{}


%style perso et macro
\definecolor{mygray1}{HTML}{e7e7e7}%light
\definecolor{mygray2}{HTML}{676767}%lightdark
\newcommand{\motcle}[1]{
  \uppercase{\textbf{#1}}
}

\newcommand{\pointilles}[1]{
  \begin{itemize}[label={}]
    \multido{}{#1}{\item \dotfill}
  \end{itemize}
}

\newenvironment{encadre}{\begin{center}\begin{minipage}[c]{0.8\linewidth}\begin{framed}}{\end{framed}\end{minipage}\end{center}}

%\newcommand{\textgreek}[1]{\foreignlanguage{greek}{#1}}

